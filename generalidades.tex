\begin{center}
    \Large {PROYECTO DE INVESTIGACIÓN PARA TRABAJO DE GRADUACIÓN \\
    \vskip 0.2cm
     ESCUELA PROFESIONAL DE INFORMÁTICA}
\end{center}
\vskip 1cm

\section{GENERALIDADES}
El internet de las cosas(IoT) ha tenido gran acogimiento a lo largo de los años, durante los cuales la cantidad de dispositivos con conexión a internet superó a la de las personas alrededor del mundo. Con el avance de la tecnología, grandes cantidades de objetos físicos han sido conectados a la red generando lo que se denomina como "big data", un término que expresa una gran complejidad con la cual el IoT viene lidiando en los últimos años.\par
\vskip 0.3cm
Esta situación se ve reflejada según cifras establecidas por grandes compañías relacionadas a este campo(entre ellas Cisco) las cuales pronostican un mayor crecimiento en cuanto a objetos conectados a internet en un futuro no muy lejano.\par
\vskip 0.3cm
El Internet de las Cosas es un campo del cual se viene hablando desde hace varios años atrás y se busca lograr establecer las bases para darle el mejor soporte en tecnologías de comunicación, pero también se mencionan soluciones que antes eran muy eficientes y que actualmente están siendo delegadas, como por ejemplo, las redes 4G, las cuales en los últimos años están perdiendo su poder en cuanto a la conectividad de dispositivos IoT, y están viviendo la gran necesidad de evolucionar para abordar los nuevos problemas que aquejan a esta tendencia; además, si hablamos de soluciones que antes eran muy buenas, no podemos dejar de lado a Cloud Computing, que es por excelencia una de las soluciones más usadas en temas de IoT debido a su alto poder de cómputo, pero como no todo es perfecto en este mundo, frente al campo de IoT con un alto nivel de tráfico, Cloud Computing deja entrever una de sus peores falencias, la centralización, la cual la convierte en una solución no muy sofisticada para este campo.\par
\vskip 0.3cm
Con esta problemática que aqueja a todo IoT nace Fog Computing, solución que viene a resolver el problema antes mencionado; Fog Computing plantea una arquitectura más distribuida y local, manejando un paradigma donde hay que aprovechar al máximo los dispositivos finales. \par
\vskip 0.8cm






\subsection{Título}
“Diseño de algoritmos paralelos sobre la arquitectura Fog Computing para reducir la latencia en IoT”

\subsection{Autor(es)}
Indicar apellidos y nombres de los participantes:
\begin{center}
    \begin{table}[!ht]
    \centering
    %\caption{Datos del alumno (s) investigador (es)}
        \begin{tabular}{llrrr} \toprule
        {\bf Código(s)} & {\bf Nombres y Apellidos} & {\bf Cargo en el proyecto} & {\bf Email} \\ \midrule
        10127003-14 & Andy Martín Panana Rosales & Estudiante invest. & apanana@unitru.edu.pe           \\
        10527010-14    & Cleiver Valera Flores & Estudiante invest. & cvalera@unitru.edu.pe            \\ \bottomrule
        \end{tabular}
    \end{table}
\end{center}

\subsection{Tipo de investigación}
    \subsubsection{De acuerdo al fin que se persigue:} 
    Aplicada
                    
    \subsubsection{De acuerdo al diseño de investigación:} 
    Explicativa

\subsection{Área y línea de Investigación}
    \subsubsection{Área de investigación :} 
    Computación paralela y distribuida
    
    \subsubsection{Línea de Investigación:} 
    Sistemas distribuidos
                
    \subsubsection{Tema de investigación :}
    Computación paralela y distribuida


\subsection{Localidad e Institución donde se desarrollará el proyecto }
  
    \subsubsection{Localidad (Dirección, Distrito, Provincia, Departamento) :} 
    Av.Juan Pablo II, Trujillo, Trujillo, La Libertad

    \subsubsection{Institución (Universidad/Facultad/Departamento):}
    Universidad Nacional de Trujillo

    \subsection{Duración del trabajo de graduación (Plan TG y desarrollo del TG)}
    Del 20/08/2018 al 07/12/2018 (4 meses)


\newpage