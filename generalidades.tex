\begin{center}
    \Large {PROYECTO DE INVESTIGACIÓN PARA TRABAJO DE GRADUACIÓN \\
    \vskip 0.2cm
     ESCUELA PROFESIONAL DE INFORMÁTICA}
\end{center}
\vskip 1cm

\section{GENERALIDADES}
El Internet de las cosas(IoT) ha tenido gran acogimiento a lo largo de los años, durante los cuales la cantidad de dispositivos con conexión a internet superó a la de las personas alrededor del mundo. Con el avance de la tecnología, grandes cantidades de objetos físicos han sido conectados a la red generando lo que se denomina como "big data", un término que expresa una gran complejidad con la cual el IoT viene lidiando en los últimos años.\par
\vskip 0.3cm
Esta situación se ve reflejada según cifras establecidas por grandes compañías relacionadas a este campo, entre ellas Cisco, las cuales pronostican un mayor crecimiento en cuanto a objetos conectados a internet en un futuro no muy lejano.\par
\vskip 0.3cm
El Internet de las Cosas es un campo del cual se viene hablando desde hace varios años atrás y se busca lograr establecer las bases para darle el mejor soporte en tecnologías de comunicación, pero también se mencionan soluciones que antes eran muy eficientes y que actualmente están siendo delegadas, como por ejemplo, las redes 4G, las cuales en los últimos años están perdiendo su poder en cuanto a la conectividad de dispositivos IoT, y están viviendo la gran necesidad de evolucionar para abordar los nuevos problemas que aquejan a esta tendencia; además, si se mencionan soluciones que antes eran muy buenas, no se puede dejar de lado a Cloud Computing, que es por excelencia una de las soluciones más usadas en temas de IoT debido a su alto poder de cómputo, pero como no todo es perfecto en este mundo, frente al campo de IoT con un alto nivel de tráfico, Cloud Computing deja entrever una de sus peores falencias, la centralización, la cual la convierte en una solución no muy sofisticada para este campo.\par
\vskip 0.3cm
Con esta problemática que aqueja a todo IoT nace Fog Computing, solución que viene a resolver el problema antes mencionado; Fog Computing plantea una arquitectura más distribuida y local, manejando un paradigma donde hay que aprovechar al máximo los dispositivos finales.\par
\vskip 0.3cm
Para comenzar a dilucidar esta interrogante es necesario ver primero el panorama actual del Internet de las Cosas, el cual ha tenido grandes avances en los últimos años gracias a la constante mejora en el desarrollo en RFID, sensores inteligentes, tecnologías de comunicación y protocolos de Internet \citep{alfuqaha2015,borgia2014}. Además, las mejoras en hardware y software de cómputo, en diversos dispositivos como los de sistemas integrados, de red, de visualización, de control, etc., han respaldado enormemente a IoT para crecer de manera lenta pero a la vez constante, logrando superar grandes expectativas y posicionarse como una gran tecnología emergente \citep{vyas2017}.\par
\vskip 0.3cm
Con el cálculo(operaciones matemáticas), la conectividad y el almacenamiento de datos cada vez más avanzados y universales, ha habido una explosión de soluciones de aplicación basadas en IoT en dominios diversificados desde la atención médica hasta la seguridad pública, desde la programación de la línea de ensamblaje hasta la fabricación y otros dominios tecnológicos como el industrial y ciudades inteligentes \citep{borgia2014,vyas2017}.\par
\vskip 0.3cm
IoT es una tecnología que está constantemente evolucionando gracias a la convergencia de otras como: comunicación inalámbrica, MEMS (sistemas microelectromecánicos), red inalámbrica de sensores, comunicación móvil, etc. \citep{vyas2017}. En particular sobre la comunicación inalámbrica se comienza a ver un nuevo horizonte para IoT, donde casi no existen problemas que no podían ser solucionados al emplear como comunicaciones de este tipo a las redes 4G y se empieza a ver con buenos ojos a la nueva y reciente mejora de dicha red, las redes 5G, las cuales se espera que aporten a la expansión del Internet de las Cosas, ayudando a impulsar el funcionamiento de las redes celulares, la seguridad de IoT, los desafíos de la red, y llevar el futuro de Internet al límite \citep{li2018}.\par
\vskip 0.3cm
Sin embargo, como toda tendencia, siempre se presentan nuevos retos que impiden un correcto desenvolvimiento. Algunos problemas que obstaculizan este paradigma son mencionados en \citep{borgia2014,ibrahim2015,mahumd2018}, estos incluyen: un incremento masivo de los datos, el aumento en la latencia y la heterogeneidad de los dispositivos.\par
\vskip 0.3cm
Cloud computing \citep{ibrahim2015} es considerada actualmente como “la piedra base” para el Internet de las cosas, en la cual muchas aplicaciones científicas han sido desarrolladas; pero esta posee una arquitectura centralizada, por ello en \citep{mahumd2018,patel2018,yi2015} se menciona a fog computing, que sería una arquitectura ideal para IoT. Particularmente, \citep{mahumd2018} menciona una interoperabilidad entre Cloud y Fog, y utilizando el simulador iFogSim se demuestran mejoras en cuanto al costo, latencia y el uso de energía, cubriendo las soluciones a ciertos problemas que presenta actualmente IoT y mejorando lo que se tiene actualmente.\par
\vskip 0.3cm
Además de ello surge también la idea de Mist Computing \citep{uehara2017,yogi2017} que permitiría construir sistemas de IoT a gran escala. Mist computing propone ser un centro de datos para cloudlets en Field Area Networks y alojarse entre pequeñas nubes (cloudlets) y la niebla (fog).\par
\vskip 0.3cm
Actualmente se generan un promedio de 2,5 quintillones de bytes por día, de los cuales un gran porcentaje es gracias a las cosas que están conectadas a internet, donde dichos objetos son usados para capturar datos y producirlos para un futuro procesamiento. La cantidad de ellos que se necesita procesar está llegando a un nivel donde las técnicas actuales que son usadas para lidiar con estos problemas tienen demasiadas complicaciones, por lo cual IoT necesita buscar otros métodos de procesamiento de datos más eficientes y escalables; frente a esta problemática es por lo que se hace mención a una de las áreas que han surgido en las últimas décadas, la informática paralela y distribuida \citep{murazzo2017}, donde sus técnicas se pueden aprovechar oportunamente para resolver problemas a gran escala y procesar los datos provenientes del paradigma de IoT \citep{piccialli2018}.\par
\vskip 0.3cm
En vista de lo mencionado anteriormente sale a relucir la siguiente interrogante ¿De qué manera se podría reducir la elevada latencia generada por las constantes peticiones desde los dispositivos IoT al servidor?\par
\vskip 0.3cm
Ahora teniendo un conocimiento amplio de la actualidad de IoT y los problemas que tanto lo aquejan, planteamos una solución utilizando Fog Computing como arquitectura distribuida para poder descentralizar las cargas que presentaba IoT con Cloud Computing e implementando algoritmos paralelos en cada punto donde se procesa la información, esto permitiría reducir la latencia de las peticiones al servidor.\par


\subsection{Título}
“Diseño de algoritmos paralelos sobre la arquitectura Fog Computing para reducir la latencia en IoT”

\subsection{Autores}
\begin{center}
    
    \begin{table}[!ht]
        \centering
        { Tabla 1: Datos de los alumnos investigadores }\par 
        %\caption{Datos del alumno (s) investigador (es)}
        \begin{tabular}{llrrr} \toprule
        {\bf Código(s)} & {\bf Nombres y Apellidos} & {\bf Cargo en el proyecto} & {\bf Email} \\ \midrule
        10127003-14 & Andy Martín Panana Rosales & Estudiante invest. & apanana@unitru.edu.pe           \\
        10527010-14    & Cleiver Valera Flores & Estudiante invest. & cvalera@unitru.edu.pe            \\ \bottomrule
        
        \end{tabular}
        \begin{center}
            \vskip 0.2cm
            {\small{Fuente: Elaboración propia.}}
        \end{center}
    \end{table}    
\end{center}

\subsection{Tipo de investigación}
    \subsubsection{De acuerdo al fin que se persigue:} 
    Aplicada
                    
    \subsubsection{De acuerdo al alcanze de la investigación:} 
    Explicativa o causal, porque se propone una solución concebida gracias a la combinación de la Arquitectura Fog Computing y la computación paralela con el objetivo de reducir la latencia que presentan las actuales propuestas en IoT, propuestas que en su gran mayoría funcionan bajo una arquitectura centralizada, Cloud Computing.

\subsection{Área y línea de Investigación}
    \subsubsection{Área de investigación :} 
    Computación paralela y distribuida
    
    \subsubsection{Línea de Investigación:} 
    Sistemas distribuidos
                
    \subsubsection{Tema de investigación :}
    Computación paralela y distribuida


\subsection{Localidad e Institución donde se desarrollará el proyecto }
  
    \subsubsection{Localidad:} 
    Av.Juan Pablo II, Trujillo, Trujillo, La Libertad

    \subsubsection{Institución:}
    Universidad Nacional de Trujillo

\subsection{Duración del trabajo de graduación (Plan TG y desarrollo del TG)}
Del 20/08/2018 al 07/12/2018 (4 meses)

\newpage

\subsection{Cronograma del trabajo de graduación}
    \begin{table}[h!]
        \centering
        { Tabla 2: Etapas y actividades para el trabajo de graduación}\par
        \begin{tabular}{|p{3cm} |p{4cm} |p{2.2cm} |p{2.6cm} |p{2.3cm}|} \hline

        %%%%%%%    CABECERA      %%%%%%%    
        
        \textit{{\bf{Etapas}}} & \textit{{\bf{Actividades/tareas}}} & \textit{{\bf{Fecha inicio}}} & \textit{{\bf{Fecha término}}} & \textit{{\bf{Hs. semanal}}}
        \\ \hline

        %%%%%%%%    FILA 1   %%%%%%%
        % COLUMNA 1 %
        \vskip 0.15cm Preparación del plan TG &
        % COLUMNA 2 %
        \vskip 0.15cm Elaborar plan TG.
        \vskip 0.15cm \par Aprobación del plan TG por jurado.  &
        % COLUMNA 3 %
        \vskip 0.15cm 20/08/2018
        \vskip 0.15cm \par 27/08/2018 &
        % COLUMNA 4 %
        \vskip 0.15cm 26/08/2018
        \vskip 0.15cm \par 07/09/2018 &
        % COLUMNA 5 %
        \vskip 0.15cm 12 - 16
        \vskip 0.15cm \par 2
        \\ \hline

        %%%%%%%%    FILA 2   %%%%%%%
        % COLUMNA 1 %
        \vskip 0.15cm Recolección de datos &
        % COLUMNA 2 %
        \vskip 0.15cm Inv. bibliográfica.
        \par Instrumentos de medición. &
        % COLUMNA 3 %
        \vskip 0.15cm 08/09/2018
        \vskip 0.15cm \par 24/09/2018  &
        % COLUMNA 4 %
        \vskip 0.15cm 03/11/2018
        \vskip 0.15cm \par 30/09/2018 &
        % COLUMNA 5 %
        \vskip 0.15cm 12 - 16
        \vskip 0.15cm \par 12 - 16
        \\  \hline

        %%%%%%%%    FILA 3   %%%%%%%
        % COLUMNA 1 %            
        \vskip 0.15cm Análisis de datos &
        % COLUMNA 2 %
        \vskip 0.15cm Comparar datos recolectados. &
        % COLUMNA 3 %
        \vskip 0.15cm 01/10/2018 &
        % COLUMNA 4 %
        \vskip 0.15cm 14/10/2018 &
        % COLUMNA 5 %
        \vskip 0.15cm 12 - 16
        \\ \hline

        %%%%%%%%    FILA 4   %%%%%%%
        % COLUMNA 1 %
        \vskip 0.15cm Redacción del informe  &
        % COLUMNA 2 %
        \vskip 0.15cm Introducción
        \vskip 0.15cm \par Marco Teórico
        \vskip 0.15cm \par Desarrollo de la propuesta, diseño de algoritmos paralelos
        \vskip 0.15cm \par Informe I: Avance Tesis
        \vskip 0.15cm \par Avance de resultados
        \vskip 0.15cm \par Informe II: Avance Tesis.
        \vskip 0.15cm \par Desarrollo de los algoritmos paralelos
        \vskip 0.15cm \par Implementación y debugging de los algoritmos paralelos
        \vskip 0.15cm \par Resultados
        \vskip 0.15cm \par Conclusiones
        \vskip 0.15cm \par Trabajos futuros &
        % COLUMNA 3 %
        \vskip 0.15cm 15/10/2018
        \vskip 0.15cm \par 15/10/2018
        \vskip 0.15cm \par 22/10/2018
        \vskip 1.15cm \par 29/10/2018
        \vskip 0.15cm \par 04/11/2018
        \vskip 0.15cm \par 03/12/2018
        \vskip 0.75cm \par 08/12/2018
        \vskip 0.6cm \par 16/02/2019
        \vskip 1.2cm \par 19/05/2019
        \vskip 0.15cm \par 11/06/2019
        \vskip 0.15cm \par 01/07/2019 &
        % COLUMNA 4 %
        \vskip 0.15cm 21/10/2018
        \vskip 0.15cm \par 21/10/2018
        \vskip 0.15cm \par 03/11/2018
        \vskip 1.15cm \par 03/11/2018
        \vskip 0.15cm \par 02/12/2018
        \vskip 0.15cm \par 07/12/2018
        \vskip 0.75cm \par 16/02/2019
        \vskip 0.6cm \par 18/05/2019
        \vskip 1.2cm \par 10/06/2019
        \vskip 0.15cm \par 30/06/2019
        \vskip 0.15cm \par 13/07/2019 &
        % COLUMNA 5 %
        \vskip 0.15cm 12 - 16
        \vskip 0.15cm \par 12 - 16
        \vskip 0.15cm \par 12 - 16
        \vskip 1.15cm \par 4
        \vskip 0.15cm \par 8
        \vskip 0.15cm \par 4  
        \vskip 0.75cm \par 12 - 16
        \vskip 0.6cm \par 12 - 16
        \vskip 1.2cm \par 12 - 16
        \vskip 0.15cm \par 12 - 16
        \vskip 0.15cm \par 12 - 16
        \\  \hline

        %%%%%%%%    FILA 5   %%%%%%%
        % COLUMNA 1 %            
        \vskip 0.15cm Sustentación de tesis \vskip 0.05cm&
        % COLUMNA 2 %
        \vskip 0.15cm Sustentacion del trabajo de investigación. &
        % COLUMNA 3 %
        \vskip 0.15cm 15/07/2019 &
        % COLUMNA 4 %
        \vskip 0.15cm 26/07/2019 &
        % COLUMNA 5 %
        \vskip 0.15cm 4
        \\ \hline

        \end{tabular}
        \begin{center}
            \vskip -0.2cm
            {\small{Fuente: Elaboración propia.}}
        \end{center}
    \end{table}

\newpage

\subsection{Recursos disponibles}
    \subsubsection{Personal}
        \begin{table}[h!]
            \centering
            { Tabla 3: Recursos disponibles - Personal}\par
            \begin{tabular}{|p{3cm}|p{3cm}|p{5cm}|} \hline
                 
            %%%%%%%    CABECERA      %%%%%%%    
            
            \textit{{\bf{Código}}} &
            \textit{{\bf{Descripción}}} &
            \textit{{\bf{Nombre}}}
            \\ \hline

            %%%%%%%%    FILA 1   %%%%%%%
            % COLUMNA 1 %
            2.3.2.7.2.2 &
            % COLUMNA 2 %
            Asesor &
            % COLUMNA 3 %
            José Luis Peralta Luján
            \\ \hline

            %%%%%%%%    FILA 2   %%%%%%%
            % COLUMNA 1 %
            2.3.2.7.2.5 &
            % COLUMNA 2 %
            Investigador &
            % COLUMNA 3 %
            Andy Martín Panana Rosales
            \\ \hline

            %%%%%%%%    FILA 3   %%%%%%%
            % COLUMNA 1 %
            2.3.2.7.2.5 &
            % COLUMNA 2 %
            Investigador &
            % COLUMNA 3 %
            Cleiver Valera Flores
            \\ \hline

            \end{tabular}
            \begin{center}
                \vskip -0.2cm
                {\small{Fuente: Elaboración propia.}}
            \end{center}
        \end{table}
    \subsubsection{Materiales y equipos}
        \subsubsubsection{Materiales de Consumo}
            \begin{table}[h!]
                \centering
                { Tabla 4: Recursos disponibles - Material de consumo}\par
                \begin{tabular}{|p{3cm}|p{2cm}|p{5cm}|} \hline
                    
                %%%%%%%    CABECERA      %%%%%%%    
                
                \textit{{\bf{Código}}} &
                \textit{{\bf{Cantidad}}} &
                \textit{{\bf{Descripción}}}
                \\ \hline

                %%%%%%%%    FILA 1   %%%%%%%
                % COLUMNA 1 %
                2.3.1.5.1.1 &
                % COLUMNA 2 %
                02 &
                % COLUMNA 3 %
                Memoria USB 8 GB
                \\ \hline

                %%%%%%%%    FILA 2   %%%%%%%
                % COLUMNA 1 %
                2.3.1.5.1.1 &
                % COLUMNA 2 %
                04 &
                % COLUMNA 3 %
                Lapicero
                \\ \hline

                %%%%%%%%    FILA 3   %%%%%%%
                % COLUMNA 1 %
                2.3.1.5.1.1 &
                % COLUMNA 2 %
                08 &
                % COLUMNA 3 %
                Fólfer Manila
                \\ \hline

                %%%%%%%%    FILA 4   %%%%%%%
                % COLUMNA 1 %
                2.3.1.5.1.1 &
                % COLUMNA 2 %
                04 &
                % COLUMNA 3 %
                Lápiz
                \\ \hline

                %%%%%%%%    FILA 5   %%%%%%%
                % COLUMNA 1 %
                2.3.1.5.1.1 &
                % COLUMNA 2 %
                02 &
                % COLUMNA 3 %
                Borrador
                \\ \hline

                %%%%%%%%    FILA 6   %%%%%%%
                % COLUMNA 1 %
                2.3.1.5.1.1 &
                % COLUMNA 2 %
                1000 &
                % COLUMNA 3 %
                Papel Bond A4
                \\ \hline

                \end{tabular}
                \begin{center}
                    \vskip -0.2cm
                    {\small{Fuente: Elaboración propia.}}
                \end{center}
            \end{table}
        \subsubsubsection{Hardware}
            \begin{table}[h!]
                \centering
                { Tabla 5: Recursos disponibles - Hardware}\par
                \begin{tabular}{|p{3cm}|p{2cm}|p{5cm}|} \hline
                    
                %%%%%%%    CABECERA      %%%%%%%    
                
                \textit{{\bf{Código}}} &
                \textit{{\bf{Cantidad}}} &
                \textit{{\bf{Descripción}}}
                \\ \hline

                %%%%%%%%    FILA 1   %%%%%%%
                % COLUMNA 1 %
                2.6.3.2.1.1 &
                % COLUMNA 2 %
                01 &
                % COLUMNA 3 %
                Raspberry Pi 3 B
                \\ \hline

                %%%%%%%%    FILA 2   %%%%%%%
                % COLUMNA 1 %
                2.6.3.2.1.1 &
                % COLUMNA 2 %
                01 &
                % COLUMNA 3 %
                Laptop ASUS Core i3
                \\ \hline

                %%%%%%%%    FILA 3   %%%%%%%
                % COLUMNA 1 %
                2.6.3.2.1.1&
                % COLUMNA 2 %
                01 &
                % COLUMNA 3 %
                Laptop LENOVO Z50-45 AMD A8
                \\ \hline

                \end{tabular}
                \begin{center}
                    \vskip -0.2cm
                    {\small{Fuente: Elaboración propia.}}
                \end{center}
            \end{table}

\newpage

        \subsubsubsection{Software}
            \begin{table}[h!]
                \centering
                { Tabla 6: Recursos disponibles - Software}\par
                \begin{tabular}{|p{3cm}|p{2cm}|p{5cm}|} \hline
                    
                %%%%%%%    CABECERA      %%%%%%%    
                
                \textit{{\bf{Código}}} &
                \textit{{\bf{Cantidad}}} &
                \textit{{\bf{Descripción}}}
                \\ \hline

                %%%%%%%%    FILA 1   %%%%%%%
                % COLUMNA 1 %
                2.6.6.1.3.2 &
                % COLUMNA 2 %
                02 &
                % COLUMNA 3 %
                Microsoft Office
                \\ \hline

                %%%%%%%%    FILA 2   %%%%%%%
                % COLUMNA 1 %
                2.6.6.1.3.2 &
                % COLUMNA 2 %
                01 &
                % COLUMNA 3 %
                Zorin OS
                \\ \hline

                %%%%%%%%    FILA 3   %%%%%%%
                % COLUMNA 1 %
                2.6.6.1.3.2 &
                % COLUMNA 2 %
                01 &
                % COLUMNA 3 %
                Linux Mint
                \\ \hline

                %%%%%%%%    FILA 4   %%%%%%%
                % COLUMNA 1 %
                2.6.6.1.3.2 &
                % COLUMNA 2 %
                02 &
                % COLUMNA 3 %
                Windows 10
                \\ \hline

                \end{tabular}
                \begin{center}
                    \vskip -0.2cm
                    {\small{Fuente: Elaboración propia.}}
                \end{center}
            \end{table}

        \subsubsubsection{Servicios}
        \begin{table}[h!]
            \centering
            { Tabla 7: Recursos disponibles - Servicios}\par
            \begin{tabular}{|p{3cm}|p{5cm}|} \hline
                 
            %%%%%%%    CABECERA      %%%%%%%    
            
            \textit{{\bf{Código}}} &
            \textit{{\bf{Descripción}}}
            \\ \hline

            %%%%%%%%    FILA 1   %%%%%%%
            % COLUMNA 1 %
            2.3.2.2.1.1 &
            % COLUMNA 2 %
            Energía Eléctrica
            \\ \hline

            %%%%%%%%    FILA 2   %%%%%%%
            % COLUMNA 1 %
            2.3.2.1.2.1 &
            % COLUMNA 2 %
            Transporte Público
            \\ \hline

            %%%%%%%%    FILA 3   %%%%%%%
            % COLUMNA 1 %
            2.3.2.2.2.3 &
            % COLUMNA 2 %
            Internet Movistar
            \\ \hline

            \end{tabular}
            \begin{center}
                \vskip -0.2cm
                {\small{Fuente: Elaboración propia.}}
            \end{center}
        \end{table}



%    \subsubsection{Locales}

%    Av.Juan Pablo II, Trujillo, Trujillo, La Libertad

\subsection{Presupuesto}
    \subsubsection{Personal}
        \begin{table}[h!]
            \centering
            { Tabla 8: Presupuesto - Personal}\par
            \begin{tabular}{|p{3cm}|p{3cm}|p{3cm}|} \hline
                 
            %%%%%%%    CABECERA      %%%%%%%    
            
            \textit{{\bf{Código}}} &
            \textit{{\bf{Descripción}}} &
            \textit{{\bf{Costo estimado}}}
            \\ \hline

            %%%%%%%%    FILA 1   %%%%%%%
            % COLUMNA 1 %
            2.3.2.7.2.2 &
            % COLUMNA 2 %
            Asesor &
            % COLUMNA 3 
            % COLUMNA 3 %%
            S/. 0.00
            \\ \hline

            %%%%%%%%    FILA 2   %%%%%%%
            % COLUMNA 1 %
            2.3.2.7.2.5 &
            % COLUMNA 2 %
            Investigador &
            % COLUMNA 3 %
            S/. 0.00
            \\ \hline

            %%%%%%%%    FILA 3   %%%%%%%
            % COLUMNA 1 %
            2.3.2.7.2.5 &
            % COLUMNA 2 %
            Investigador &
            % COLUMNA 3 %
            S/. 0.00
            \\ \hline

            %%%%%%%%    FILA 4   %%%%%%%
            % COLUMNA 1 %
             &
            % COLUMNA 2 %
            \bf{Total} &
            % COLUMNA 3 %
            S/. 0.00
            \\ \hline

                
            \end{tabular}
            \begin{center}
                    \vskip -0.2cm
                    {\small{Fuente: Elaboración propia.}}
                \end{center}
        \end{table}

\newpage

    \subsubsection{Materiales y equipos}
        \subsubsubsection{Materiales de Consumo}
            \begin{table}[h!]
                \centering
                { Tabla 9: Presupuesto - Materiales de Consumo}\par
                \begin{tabular}{|p{2.2cm}|p{1.8cm}|p{3.8cm}|p{2.2cm}|} \hline
                    
                %%%%%%%    CABECERA      %%%%%%%    
                
                \textit{{\bf{Código}}} &
                \textit{{\bf{Cantidad}}} &
                \textit{{\bf{Descripción}}} &
                \textit{{\bf{Costo}}}
                \\ \hline

                %%%%%%%%    FILA 1   %%%%%%%
                % COLUMNA 1 %
                2.3.1.5.1.1 &
                % COLUMNA 2 %
                02 &
                % COLUMNA 3 %
                Memoria USB 8 GB &
                % COLUMNA 4 %
                S/. 50.00
                \\ \hline

                %%%%%%%%    FILA 2   %%%%%%%
                % COLUMNA 1 %
                2.3.1.5.1.1 &
                % COLUMNA 2 %
                04 &
                % COLUMNA 3 %
                Lapicero &
                % COLUMNA 4 %
                S/. 8.00
                \\ \hline

                %%%%%%%%    FILA 3   %%%%%%%
                % COLUMNA 1 %
                2.3.1.5.1.1 &
                % COLUMNA 2 %
                08 &
                % COLUMNA 3 %
                Fólfer Manila &
                % COLUMNA 4 %
                S/. 4.00
                \\ \hline

                %%%%%%%%    FILA 4   %%%%%%%
                % COLUMNA 1 %
                2.3.1.5.1.1 &
                % COLUMNA 2 %
                04 &
                % COLUMNA 3 %
                Lápiz &
                % COLUMNA 4 %
                S/. 3.00
                \\ \hline

                %%%%%%%%    FILA 5   %%%%%%%
                % COLUMNA 1 %
                2.3.1.5.1.1 &
                % COLUMNA 2 %
                02 &
                % COLUMNA 3 %
                Borrador &
                % COLUMNA 4 %
                S/. 1.00
                \\ \hline

                %%%%%%%%    FILA 6   %%%%%%%
                % COLUMNA 1 %
                2.3.1.5.1.1 &
                % COLUMNA 2 %
                1000 &
                % COLUMNA 3 %
                Papel Bond A4 &
                % COLUMNA 4 %
                S/. 22.00
                \\ \hline

                %%%%%%%%    FILA 7   %%%%%%%
                % COLUMNA 1 %
                &
                % COLUMNA 2 %
                &
                % COLUMNA 3 %
                \bf{Total} &
                % COLUMNA 4 %
                S/. 88.00
                \\ \hline
                \end{tabular}
                \begin{center}
                    \vskip -0.2cm
                    {\small{Fuente: Elaboración propia.}}
                \end{center}
            \end{table}

        \subsubsubsection{Hardware}
            \begin{table}[h!]
                \centering
                { Tabla 10: Presupuesto - Hardware}\par
                \begin{tabular}{|p{2cm}|p{1.8cm}|p{3.8cm}|p{1.8cm}|p{2cm}|} \hline
                    
                %%%%%%%    CABECERA      %%%%%%%    
                
                \textit{{\bf{Código}}} &
                \textit{{\bf{Cantidad}}} &
                \textit{{\bf{Descripción}}} &
                \textit{{\bf{Vida útil}}}  &
                \textit{{\bf{Costo}}}
                \\ \hline

                %%%%%%%%    FILA 1   %%%%%%%
                % COLUMNA 1 %
                2.6.3.2.1.1 &
                % COLUMNA 2 %
                01 &
                % COLUMNA 3 %
                Raspberry Pi 3 B &
                % COLUMNA 4 %
                2 años &
                % COLUMNA 5 %
                S/. 230.00
                \\ \hline

                %%%%%%%%    FILA 2   %%%%%%%
                % COLUMNA 1 %
                2.6.3.2.1.1 &
                % COLUMNA 2 %
                01 &
                % COLUMNA 3 %
                Laptop ASUS Core i3 &
                % COLUMNA 4 %
                3 años &
                % COLUMNA 5 %
                S/. 1500.00
                \\ \hline

                %%%%%%%%    FILA 3   %%%%%%%
                % COLUMNA 1 %
                2.6.3.2.1.1&
                % COLUMNA 2 %
                01 &
                % COLUMNA 3 %
                Laptop LENOVO Z50-70 AMD A8 &
                % COLUMNA 4 %
                3 años &
                % COLUMNA 5 %
                S/. 1800.00
                \\ \hline

                %%%%%%%%    FILA 4   %%%%%%%
                % COLUMNA 1 %
                &
                % COLUMNA 2 %
                &
                % COLUMNA 3 %
                &
                % COLUMNA 4 %
                \bf{Total} &
                % COLUMNA 5 %
                S/. 3530.00
                \\ \hline

                \end{tabular}
                \begin{center}
                    \vskip -0.2cm
                    {\small{Fuente: Elaboración propia.}}
                \end{center}
            \end{table}

        \subsubsubsection{Software}
            \begin{table}[h!]
                \centering
                { Tabla 11: Presupuesto - Software}\par
                \begin{tabular}{|p{2.2cm}|p{1.8cm}|p{3.8cm}|p{2.2cm}|} \hline
                    
                %%%%%%%    CABECERA      %%%%%%%    
                
                \textit{{\bf{Código}}} &
                \textit{{\bf{Cantidad}}} &
                \textit{{\bf{Descripción}}} &
                \textit{{\bf{Costo}}}
                \\ \hline

                %%%%%%%%    FILA 1   %%%%%%%
                % COLUMNA 1 %
                2.6.6.1.3.2 &
                % COLUMNA 2 %
                02 &
                % COLUMNA 3 %
                Microsoft Office &
                % COLUMNA 4 %
                S/. 365.00
                \\ \hline

                %%%%%%%%    FILA 2   %%%%%%%
                % COLUMNA 1 %
                2.6.6.1.3.2 &
                % COLUMNA 2 %
                01 &
                % COLUMNA 3 %
                Zorin OS &
                % COLUMNA 4 %
                S/. 0.00
                \\ \hline

                %%%%%%%%    FILA 3   %%%%%%%
                % COLUMNA 1 %
                2.6.6.1.3.2 &
                % COLUMNA 2 %
                01 &
                % COLUMNA 3 %
                Linux Mint &
                % COLUMNA 4 %
                S/. 0.00
                \\ \hline

                %%%%%%%%    FILA 4   %%%%%%%
                % COLUMNA 1 %
                2.6.6.1.3.2 &
                % COLUMNA 2 %
                02 &
                % COLUMNA 3 %
                Windows 10 &
                % COLUMNA 4 %
                S/. 555.00
                \\ \hline

                %%%%%%%%    FILA 5   %%%%%%%
                % COLUMNA 1 %
                &
                % COLUMNA 2 %
                &
                % COLUMNA 3 %
                \bf{Total} &
                % COLUMNA 4 %
                S/. 920.00
                \\ \hline

                \end{tabular}
                \begin{center}
                    \vskip -0.2cm
                    {\small{Fuente: Elaboración propia.}}
                \end{center}
            \end{table}

\newpage

        \subsubsubsection{Servicios}
            \begin{table}[h!]
                \centering
                { Tabla 12: Presupuesto- Servicios}\par
                \begin{tabular}{|p{2cm}|p{4cm}|p{2.2cm}|} \hline
                    
                %%%%%%%    CABECERA      %%%%%%%    
                
                \textit{{\bf{Código}}} &
                \textit{{\bf{Descripción}}} &
                \textit{{\bf{Costo}}}
                \\ \hline

                %%%%%%%%    FILA 1   %%%%%%%
                % COLUMNA 1 %
                2.3.2.2.1.1 &
                % COLUMNA 2 %
                Energía Eléctrica &
                % COLUMNA 3 %
                S/. 288.00
                \\ \hline

                %%%%%%%%    FILA 2   %%%%%%%
                % COLUMNA 1 %
                2.3.2.1.2.1 &
                % COLUMNA 2 %
                Transporte Público &
                % COLUMNA 3 %
                S/. 150.00
                \\ \hline

                %%%%%%%%    FILA 3   %%%%%%%
                % COLUMNA 1 %
                2.3.2.2.2.3 &
                % COLUMNA 2 %
                Internet Movistar &
                % COLUMNA 3 %
                S/. 480.00
                \\ \hline

                %%%%%%%%    FILA 4   %%%%%%%
                % COLUMNA 1 %
                &
                % COLUMNA 2 %
                \bf{Total} &
                % COLUMNA 3 %
                S/. 918.00
                \\ \hline

                \end{tabular}
                \begin{center}
                    \vskip -0.2cm
                    {\small{Fuente: Elaboración propia.}}
                \end{center}
            \end{table}
    \subsubsection{Costo del Proyecto}
        \begin{table}[h!]
            \centering
            { Tabla 13: Costo del Proyecto}\par
            \begin{tabular}{|p{5cm}|p{4cm}|} \hline

                %%%%%%%    CABECERA      %%%%%%%    

                \textit{{\bf{Descripción}}} &
                \textit{{\bf{Costo}}}
                \\ \hline

                %%%%%%%%    FILA 1   %%%%%%%
                % COLUMNA 1 %
                Personal &
                % COLUMNA 2 %
                S/. 0.00
                \\ \hline

                %%%%%%%%    FILA 2   %%%%%%%
                % COLUMNA 1 %
                Materiales de consumo &
                % COLUMNA 2 %
                S/. 88.00
                \\ \hline

                %%%%%%%%    FILA 3   %%%%%%%
                % COLUMNA 1 %
                Hardware &
                % COLUMNA 2 %
                S/. 3530.00
                \\ \hline

                %%%%%%%%    FILA 4   %%%%%%%
                % COLUMNA 1 %
                Software &
                % COLUMNA 2 %
                S/. 920.00
                \\ \hline

                %%%%%%%%    FILA 5   %%%%%%%
                % COLUMNA 1 %
                Servicios &
                % COLUMNA 2 %
                S/. 918.00
                \\ \hline

                %%%%%%%%    FILA 6   %%%%%%%
                % COLUMNA 1 %
                \bf{Total} &
                % COLUMNA 2 %
                S/. 5456.00
                \\ \hline

            \end{tabular}
            \begin{center}
                \vskip -0.2cm
                {\small{Fuente: Elaboración propia.}}
            \end{center}
        \end{table}

\subsection{Financiamiento}
    \subsubsection{Con recursos universitarios}
        Ninguno
    \subsubsection{Con recursos externos}
        Ninguno
    \subsubsection{Autofinanciación}
        Autofinanciado
